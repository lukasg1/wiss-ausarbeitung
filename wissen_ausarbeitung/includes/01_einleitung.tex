\chapter{Einleitung}
\section{Motivation}
Bei der Einführung neuer Systeme in einem unternehmerischen oder universitären
Kontext ist die Schulung der Endbenutzerinnen und -benutzer ein zentraler Erfolgsfaktor.
\section{Problemstellung}
\section{Zielsetzung}
\section{Vorgehensweise}
\section{Aufbau der Arbeit}


% Die Einführung neuer Systeme und Prozesse erfordert
% stets eine umsichtige und durchdachte Herangehensweise. Diese beginnt bereits
% bei der Systemplanung und erstreckt sich über die Entwicklung bis hin zur abschließenden Migration.
% \footcite[Vgl.][S. 882 ff.]{laudonWirtschaftsinformatikEinfuehrung2016}
% Oftmals kommt es hierbei in der praktischen Umsetzung jedoch zu Fehlern, durch welche die geplante Einführung
% des jeweiligen Systems schlussendlich entweder stark verzögert oder sogar gar nicht stattfindet. Als Beispiel
% hierfür lässt sich die Implementierung eines neuen „\ac{ERP}“-Systems anführen. Auf einer Konferenz für \ac{ERP} benennen
% Becker und Winkelmann 2008 die fehlende Schulung von Mitarbeitenden als zentralen Faktor für das Scheitern eines solchen Projektes.
% \footcite[Vgl.][27]{becker10MoglichkeitenERPProjekt2008}
% Für die erfolgreiche Migration auf ein neues System sei es daher unabdingbar, die Endbenutzerinnen und -benutzer hinsichtlich der
% Bedienung und Funktionen zu schulen, um ein Scheitern der neuen Plattform oder Anwendung und damit eine Vergeudung monetärer und zeitlicher Ressourcen zu vermeiden.  
% \footcite[Vgl.][886]{laudonWirtschaftsinformatikEinfuehrung2016}

% Auch in der Fakultät Wirtschaft an der \acs{DHBW} Stuttgart soll im Jahr 2024 ein neues System eingeführt werden.
% Konkret handelt es sich hierbei um den zentralen Raumplanungsassistenten \acs{RAPLA}, welcher Studiengangsleiterinnen
% und -leiter sowie die zugehörigen Sekretariate bei der Buchung von Räumen unterstützen soll. Bislang erfolgt die Buchung von
% Räumen in der Fakultät Wirtschaft über Excel-Tabellen, wodurch ein sehr hoher Abstimmungsaufwand unter den Studiengängen
% hinsichtlich der effizientesten Belegung von Seminarräumen besteht. Diesem Aspekt kommt insbesondere in Anbetracht einer
% allgemeinen Raumknappheit der Hochschule eine hohe Bedeutung zu. Da \acs{RAPLA} jedoch noch nicht an allen Standorten im Einsatz ist,
% fehlt bislang in vielen Studienzentren das Wissen über den Aufbau, die Beschaffenheit sowie die Funktionen der Anwendung.

% Gegenwärtig wird daher ein statisches Schulungsdokument erstellt, durch welches den jeweiligen Studienzentren sowie Angehörigen der \acs{DHBW} Stuttgart
% bei der Einführung des Raumplanungsassistenten relevante Informationen zur Verfügung gestellt werden.
% Um den Teilnehmerinnen und Teilnehmern der Schulung einen möglichst hohen
% Mehrwert und einen intuitiven Einstieg in die Anwendung zu bieten, soll als Eigenleistung im Rahmen der Projektkonzeption ergänzend zu
% der bereits bestehenden Schulungsdokumentation ein Wissensquiz entwickelt werden, welches direkt in Moodle abgebildet sein wird.
% Dieses soll die Nutzerinnen und Nutzer dazu motivieren, sich mit Anwendung aktiv zu beschäftigen und sie zum Lernen und Ausprobieren
% anregen. Den erfolgreichen Abschluss des Kurses bildet eine persönliche Zertifizierung, welche den Teilnehmenden digital
% zur Verfügung gestellt wird. Die Umsetzung in Moodle wird erreicht, indem in einem ersten Schritt eine umfassende Analyse der Gegebenheiten erfolgt, zu denen bspw.
% die Ziel- und Interessengruppen, vor- und nachgelagerte Prozesse sowie das Programm selbst gehören. Zusätzlich werden die gegenwärtigen
% Schulungsunterlagen gesichtet und die Funktionsfähigkeit von \acs{RAPLA} mittels eines Testzugangs untersucht. Darauf aufbauend wird
% in einem zweiten Schritt eine Seite in Moodle erstellt, auf welcher das Wissensquiz abgebildet sein wird. Im dritten
% Schritt erfolgt eine Evaluation des neuen Schulungsprozesses, welcher eine empirische Validierung mit echten Anwenderinnen und
% Anwendern vorausgeht. Anhand dieser Evaluation soll abschließend hervorgehen, ob und inwiefern die Moodle-Zertifizierung vorteilhaft
% für den operativen Einsatz ist und welche Änderungen und Ergänzungen darauf aufbauend für zukünftige Aktualisierungen in Betracht zu ziehen sind.

% ----


% Daher soll im Rahmen der Studienfächer „Projektkonzeption“, „Projekt“ und „Integrationsseminar“ ein Schulungskonzept erstellt
% werden, welches die jeweiligen Studienzentren sowie Angehörigen der DHBW Stuttgart bei der erfolgreichen Einführung von \acs{RAPLA}
% unterstützt. Dies wird erreicht, indem in einem ersten Schritt eine umfassende Analyse der Gegebenheiten erfolgt, zu denen bspw.
% die Ziel- und Interessengruppen, vor- und nachgelagerte Prozesse sowie das Programm selbst gehören. Zusätzlich werden die gegenwärtigen
% Schulungsunterlagen gesichtet und die Funktionsfähigkeit von \acs{RAPLA} mittels eines Testzugangs untersucht. Darauf aufbauend wird
% in einem zweiten Schritt ein vollumfängliches Schulungskonzept inklusive Schulungsunterlagen erstellt, das Auskunft darüber gibt,
% welche Personenkreise zu welchem Zeitpunkt mit welchen Mitteln zu schulen sind, um das jeweilige Ziel zu erreichen. Im dritten
% Schritt erfolgt eine Evaluation des neuen Schulungsprozesses, welcher eine empirische Validierung mit echten Anwenderinnen und
% Anwendern vorausgeht. Anhand dieser Evaluation soll abschließend hervorgehen, ob und inwiefern das neue Konzept vorteilhaft
% für den operativen Einsatz ist und welche neuen Funktionen darauf aufbauend für zukünftige Aktualisierungen in Betracht zu ziehen sind.

