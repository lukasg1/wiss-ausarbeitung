\chapter{Einleitung}
\section{Motivation}\label{sec:motivation}
Bei der Einführung neuer Systeme in einem unternehmerischen oder universitären
Kontext ist neben einer strukturierten und umfassenden Anforderungsanalyse
auch
die Schulung der Endbenutzerinnen und -benutzer ein zentraler Erfolgsfaktor.
\footcite[Vgl.][S. 189 ff.]{leeEmpiricalStudyRelationships1995}
\dots
Die Integration von Wissensquizzen in Schulungskonezpte wird dabei
als eine geeignete Methode angesehen, um die Lernmotivation der Teilnehmenden zu steigern.
\footcites[Vgl.][83]{huangEmpoweringEndUsers1998}[1080]{maurerEQuizSimpleTool2007}[147]{paaERPEndUserTrainingELearning2014}
Darüber hinaus kann durch die Analyse von Larsen u. a. 2015 belegt werden, dass
die vermittelten Lerninhalte durch den Einsatz von Wissensquizzen besser verinnerlicht werden.
\footcite[Vgl.][S. 748 ff.]{larsenEffectsTestenhancedLearning2015}
Hierbei stellen die Autorinnen und Autoren fest, dass insbesondere die Teilnehmenden, welche
sich wiederholt dem Wissensquiz unterziehen, eine um elf Prozentpunkte höhere Wissensretention
aufweisen im Vergleich zu denjenigen, welche lediglich wiederholt Schulungsunterlagen studieren.
\footcite[Vgl.][748]{larsenEffectsTestenhancedLearning2015}
\section{Problemstellung}
Wie in \textit{Kapitel \ref{sec:motivation}} dargelegt, kann die Integration von Wissensquizzen
generell einen positiven Effekt auf den Lernerfolg haben. Voraussetzung hierfür ist jedoch, dass
bereits im Vorfeld Schulungsunterlagen und Handbücher vorliegen, auf dessen Grundlage
die entsprechenden Fragen erstellt werden können. Dies ist bspw. im Kontext
der Einführung des zentralen Raumplanungsassistenten \ac{RAPLA} an der \ac{DHBW} Stuttgart
gegeben. Hierfür existieren bereits Schulungsentwürfe, welche
die entsprechenden Sekretariate auf die Nutzung des Systems vorbereiten sollen.
Zeitgleich ist allerdings auch festzustellen, dass bislang noch keine
konkreten Konzepte für die Integration von Wissensquizzen in das Schulungskonzept
existieren. Zudem lässt sich der Aufwand für die Erstellung eines solchen
Wissensquizzes vor dem Hintergrund, dass \ac{RAPLA} in nahezu
identischer Form künftig flächendeckend
eingeführt werden soll, relativieren.


\section{Zielsetzung}\label{sec:zielsetzung}
Auf Basis der genannten Problemstellung soll im Rahmen dieser Arbeit
die Konzeption eines Wissensquiz für den zentralen \acp{RAPLA} der \acp{DHBW}
Stuttgart in Moodle erfolgen.
Hierfür wird im Laufe des Projektverlaufs ein Fragenkatalog mit ungefähr 30
theoretischen und praxisnahen Aufgaben erstellt werden, wobei diese zielgruppenorientiert
und an die vorhandenen Schulungsunterlagen angelehnt sein sollen.
Die erfolgreiche Teilnahme am Wissensquiz soll durch eine persönliche Zertifizierung
bescheinigt werden, welche den Teilnehmenden digital zur Verfügung gestellt wird.
Als Plattform für die Umsetzung ist das webbasierte Lernmanagementsystem Moodle
vorgesehen.
Ziel des Projektes ist es, eine technisch und inhaltlich einwandfreie Umsetzung
des Wissensquiz zu erreichen, welche idealerweise
zudem die Nutzerinnen und Nutzer dazu motiviert, sich mit der Anwendung aktiv
auseinanderzusetzen.
Die Erfüllung dieses Ziels wird am Ende sowohl durch eine Expertin oder einen
Experten für \ac{RAPLA}, als auch durch die Zielgruppe der Sekretariate selbst
sichergestellt werden.
\section{Methodik und Vorgehensweise}
Zur Erreichung des in \textit{Kapitel \ref{sec:zielsetzung}} definierten Ziels werden
im theoretischen Teil dieser Arbeit relevante didaktische Konzepte für die Erstellung von Wissensquizzen
diskutiert. Methodisch wird hierbei nach dem Schemaa einer \ac{SLA} nach Kitchenham u. a. 2007 vorgegangen.
Insgesamt wird hierbei in vier Schritten vorgegangen:
Zuerst erfolgt eine Analyse der vorhandenen Schulungsunterlagen für \ac{RAPLA}.
Maßgeblich hierfür ist insbesondere das Grundlagenhandbuch, welches sich an die Zielgruppe der
Endanwenderinnen und -anwender richtet. Auf Basis dieser Analyse wird im zweiten Schritt
ein Fragenkatalog erstellt, welcher die Grundlage für das Wissensquiz bildet.
Dieser Fragenkatalog wird durch praxisnahe Aufgabenstellungen, welche ebenfalls
in der späteren Schulung vorkommen, ergänzt. Im dritten Schritt erfolgt die Umsetzung
des Wissensquiz in Moodle. Hierbei wird insbesondere auf eine strukturell
sinnvolle Anordnung der Fragen sowie eine technisch einwandfreie Umsetzung geachtet.
Abschließend erfolgt im vierten Schritt die Erprobung des Wissensquiz durch die Zielgruppe.
Hierfür sind vorab klare Kriterien zu definieren, anhand derer die Erprobungsergebnisse
ausgewertet werden können. Über den ganzen Erprobungsprozess hinweg wird dafür
ein Protokoll geführt. Die Ergebnisse der Erprobung werden für den fünften
und letzten Schritt genutzt, welcher die Ableitung von Optimierungsmaßnahmen
zum Ziel hat.
\section{Aufbau der Arbeit}
% ändern und an diese Arbeit anpassen
Die vorliegende Arbeit ist wie folgt aufgebaut: Das erste Kapitel
dieser Arbeit dient der Einleitung in die Thematik und soll ebenso
Relevanz und Aktualität des Themas aufzeigen. Zusätzlich enthält dieses
Kapitel die Problemstellung und Zielsetzung der Arbeit. Das zweite Kapitel
dient der Darlegung des theoretischen Hintergrunds in Bezug auf
E-Learning und aktuelle didaktische Konzepte
für die Erstellung von Wissensquizzen. Methodisch wird hierbei
nach dem Schema einer \acp{SLA} vorgegangen. Diese Konzepte werden
miteinander verglichen, um auf diese Weise eine Ausgangsbasis
für den vorliegenden Anwendungsfall zu schaffen.
Im dritten Kapitel wird in direkter Anknüpfung Bezug auf das vorliegende Projekt
genommen, indem der Umfang, der Gegenstand und die Anforderungen an die Wissensquizerstellung
für \ac{RAPLA} erläutert werden.
Im vierten Kapitel wird der erste Konzeptentwurf für das Wissensquiz
dargestellt. Hierbei wird insbesondere thematisiert, in welcher Form
die vorhandenen Unterlagen einer Analyse unterzogen werden. Darauf aufbauend
werden basierend auf den Ergebnissen des ersten Schrittes die Fragen für das Wissensquiz
erstellt. Im fünften Kapitel wird die technische Umsetzung des Wissensquiz
in Moodle beschrieben. Hierbei wird insbesondere auf die Anforderungen an die
Systemtechnik eingegangen. Im sechsten Kapitel wird die Erprobung des Wissensquiz
durch die Zielgruppe beschrieben. In diesem Falle handelt es sich um Sekretariate
der Studiengangsleitungen der Fakultät Wirtschaft an der \ac{DHBW} Stuttgart.
Auf dieser Grundlage können ebenso die Erprobungsergebnisse analysiert und
Optimierungsmaßnahmen abgeleitet werden.
Diese können insbesondere für eine weiterführende Arbeit
am Wissensquiz von Relevanz sein. Das Fazit im siebten und letzten Kapitel
dieser Arbeit fasst die Ergebnisse zusammen, reflektiert diese kritisch und gibt einen Ausblick auf
weitere mögliche Untersuchungsfelder.



% Die Einführung neuer Systeme und Prozesse erfordert
% stets eine umsichtige und durchdachte Herangehensweise. Diese beginnt bereits
% bei der Systemplanung und erstreckt sich über die Entwicklung bis hin zur abschließenden Migration.
% \footcite[Vgl.][S. 882 ff.]{laudonWirtschaftsinformatikEinfuehrung2016}
% Oftmals kommt es hierbei in der praktischen Umsetzung jedoch zu Fehlern, durch welche die geplante Einführung
% des jeweiligen Systems schlussendlich entweder stark verzögert oder sogar gar nicht stattfindet. Als Beispiel
% hierfür lässt sich die Implementierung eines neuen „\ac{ERP}“-Systems anführen. Auf einer Konferenz für \ac{ERP} benennen
% Becker und Winkelmann 2008 die fehlende Schulung von Mitarbeitenden als zentralen Faktor für das Scheitern eines solchen Projektes.
% \footcite[Vgl.][27]{becker10MoglichkeitenERPProjekt2008}
% Für die erfolgreiche Migration auf ein neues System sei es daher unabdingbar, die Endbenutzerinnen und -benutzer hinsichtlich der
% Bedienung und Funktionen zu schulen, um ein Scheitern der neuen Plattform oder Anwendung und damit eine Vergeudung monetärer und zeitlicher Ressourcen zu vermeiden.  
% \footcite[Vgl.][886]{laudonWirtschaftsinformatikEinfuehrung2016}

% Auch in der Fakultät Wirtschaft an der \acs{DHBW} Stuttgart soll im Jahr 2024 ein neues System eingeführt werden.
% Konkret handelt es sich hierbei um den zentralen Raumplanungsassistenten \acs{RAPLA}, welcher Studiengangsleiterinnen
% und -leiter sowie die zugehörigen Sekretariate bei der Buchung von Räumen unterstützen soll. Bislang erfolgt die Buchung von
% Räumen in der Fakultät Wirtschaft über Excel-Tabellen, wodurch ein sehr hoher Abstimmungsaufwand unter den Studiengängen
% hinsichtlich der effizientesten Belegung von Seminarräumen besteht. Diesem Aspekt kommt insbesondere in Anbetracht einer
% allgemeinen Raumknappheit der Hochschule eine hohe Bedeutung zu. Da \acs{RAPLA} jedoch noch nicht an allen Standorten im Einsatz ist,
% fehlt bislang in vielen Studienzentren das Wissen über den Aufbau, die Beschaffenheit sowie die Funktionen der Anwendung.

% Gegenwärtig wird daher ein statisches Schulungsdokument erstellt, durch welches den jeweiligen Studienzentren sowie Angehörigen der \acs{DHBW} Stuttgart
% bei der Einführung des Raumplanungsassistenten relevante Informationen zur Verfügung gestellt werden.
% Um den Teilnehmerinnen und Teilnehmern der Schulung einen möglichst hohen
% Mehrwert und einen intuitiven Einstieg in die Anwendung zu bieten, soll als Eigenleistung im Rahmen der Projektkonzeption ergänzend zu
% der bereits bestehenden Schulungsdokumentation ein Wissensquiz entwickelt werden, welches direkt in Moodle abgebildet sein wird.
% Dieses soll die Nutzerinnen und Nutzer dazu motivieren, sich mit Anwendung aktiv zu beschäftigen und sie zum Lernen und Ausprobieren
% anregen. Den erfolgreichen Abschluss des Kurses bildet eine persönliche Zertifizierung, welche den Teilnehmenden digital
% zur Verfügung gestellt wird. Die Umsetzung in Moodle wird erreicht, indem in einem ersten Schritt eine umfassende Analyse der Gegebenheiten erfolgt, zu denen bspw.
% die Ziel- und Interessengruppen, vor- und nachgelagerte Prozesse sowie das Programm selbst gehören. Zusätzlich werden die gegenwärtigen
% Schulungsunterlagen gesichtet und die Funktionsfähigkeit von \acs{RAPLA} mittels eines Testzugangs untersucht. Darauf aufbauend wird
% in einem zweiten Schritt eine Seite in Moodle erstellt, auf welcher das Wissensquiz abgebildet sein wird. Im dritten
% Schritt erfolgt eine Evaluation des neuen Schulungsprozesses, welcher eine empirische Validierung mit echten Anwenderinnen und
% Anwendern vorausgeht. Anhand dieser Evaluation soll abschließend hervorgehen, ob und inwiefern die Moodle-Zertifizierung vorteilhaft
% für den operativen Einsatz ist und welche Änderungen und Ergänzungen darauf aufbauend für zukünftige Aktualisierungen in Betracht zu ziehen sind.

% ----


% Daher soll im Rahmen der Studienfächer „Projektkonzeption“, „Projekt“ und „Integrationsseminar“ ein Schulungskonzept erstellt
% werden, welches die jeweiligen Studienzentren sowie Angehörigen der DHBW Stuttgart bei der erfolgreichen Einführung von \acs{RAPLA}
% unterstützt. Dies wird erreicht, indem in einem ersten Schritt eine umfassende Analyse der Gegebenheiten erfolgt, zu denen bspw.
% die Ziel- und Interessengruppen, vor- und nachgelagerte Prozesse sowie das Programm selbst gehören. Zusätzlich werden die gegenwärtigen
% Schulungsunterlagen gesichtet und die Funktionsfähigkeit von \acs{RAPLA} mittels eines Testzugangs untersucht. Darauf aufbauend wird
% in einem zweiten Schritt ein vollumfängliches Schulungskonzept inklusive Schulungsunterlagen erstellt, das Auskunft darüber gibt,
% welche Personenkreise zu welchem Zeitpunkt mit welchen Mitteln zu schulen sind, um das jeweilige Ziel zu erreichen. Im dritten
% Schritt erfolgt eine Evaluation des neuen Schulungsprozesses, welcher eine empirische Validierung mit echten Anwenderinnen und
% Anwendern vorausgeht. Anhand dieser Evaluation soll abschließend hervorgehen, ob und inwiefern das neue Konzept vorteilhaft
% für den operativen Einsatz ist und welche neuen Funktionen darauf aufbauend für zukünftige Aktualisierungen in Betracht zu ziehen sind.

