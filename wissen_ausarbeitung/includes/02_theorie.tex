\chapter{Theoretischer Hintergrund}
\section{E-Learning und digitale Wissensvermittlung}
E-Learning stellt eine neue Lernumgebung dar, die durch den Einsatz von digitalen Medien und 
Technologien die Wissensvermittlung unterstützt. Es ermöglicht den Lernenden, unabhängig von Zeit und 
Ort zu lernen und bietet eine Vielzahl von Lernmaterialien und -methoden. %(Quelle:) 
Durch den Einsatz von E-Learning können Lernende ihr Wissen effizienter und flexibler erweitern und vertiefen. %(Quelle:) 
Dieser Ansatz hat in den letzten Jahren an Bedeutung gewonnen und wird zunehmend in 
Bildungseinrichtungen und Unternehmen eingesetzt. So fördert E-Learning die Selbstorganisation, kritisches
Denken oder die Fähigkeit zur Problemlösung der Lernenden. %(Quelle: https://www.researchgate.net/profile/Moradeke-Adewumi/publication/266603233_E-Learning_and_Its_Effects_on_Teaching_and_Learning_in_a_Global_Age/links/62052191634ff774f4c0a84b/E-Learning-and-Its-Effects-on-Teaching-and-Learning-in-a-Global-Age.pdf, Seite 1 f.)
Zudem ermöglicht E-Learning eine individuelle Anpassung des Lernprozesses an die Bedürfnisse und 
\section{Didaktische Konzepte für die Wissensquiz-Erstellung}
\section{Zertifizierungen als Erfolgsfaktor}
\section{Gestaltung von Usability-Tests}