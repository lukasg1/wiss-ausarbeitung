\chapter{Theoretischer Hintergrund}
\section{E-Learning und digitale Wissensvermittlung}
\section{Didaktische Konzepte für die Wissensquiz-Erstellung}
\section{Zertifizierungen als Erfolgsfaktor}
\section{RAPLA 2.0 als Prüfungsgegenstand}
% Bevor die spezifischen Anforderungen ermittelt werden können, ist es zunächst notwendig,
% den Gegenstand des vorliegenden Projektes genauer zu definieren.
% \footcite[Vgl.][S. 270 ff.]{schuesslerEinfuehrungDokumentenManagementSystemsFuer2008}
% Hierunter fällt sowohl eine Beschreibung des gegenwärtigen Zustands als auch eine Skizze des gewünschten Endergebnisses.
% \footcite[Vgl.][39]{braatzEntwicklungMethodeZur2005a}
% Darauf aufbauend können schließlich Anforderungen abgeleitet werden, welche eine Grundlage für die weitere Planung und Umsetzung des Projektes bilden.

% Die Anforderungsbeschreibung erfolgt hierbei im Stil eines Lastenheftes, welches originär in der Industrie zur Beschreibung von Anforderungen
% an zu entwickelnde Systeme seitens des Auftraggebers verwendet wird.
% \footcites[Vgl.][60]{gilzRequirementsEngineeringUnd2014}[][1]{houdekRequirementsEngineeringErfahrungen2003}[][483]{lauberWieFormalSoll1981}
% Im Allgemeinen umfasst dieses gemäß dem \ac{DIN} „die vom Auftraggeber festgelegte Gesamtheit der Forderungen an die Lieferungen und Leistungen eines Auftragnehmers
% innerhalb eines Auftrages“.
% \footcite[Vgl.][13]{baerBegriffsbestimmungenUndDefinitionen2017}
% Da es bisher keinen branchenübergreifenden Standard oder eine allgemeine Norm für die
% Erstellung von Lasten- und Pflichtenheften gibt\footcite[Vgl.][219]{goehlichArbeitenMitAnforderungen2021},
% soll in dieser Arbeit als Schema für die Anforderungsbeschreibung das Lasten-/Pflichtenheft für den Einsatz von Automatisierungssystemen nach Richtlinie VDI/VDE 3694
% mit Änderungen verwendet werden.
% \footcite[Vgl.][S. 510 f.]{schellongEnergiemanagementsysteme2016}
% Dieses ist dadurch gekennzeichnet, dass es die Anforderungen an ein Projekt in verschiedene Bereiche unterteilt.

% Die Anforderungen an das Projekt werden hierbei - wie üblich für Projekte, die agil gestaltet sind -
% in Absprache mit anderen an dem Projekt beteiligten Personen Gegenstand stetig wiederkehrender Überprüfungen und Anpassungen sein.
% \footcites[Vgl.][1]{fazal-baqaieSkalierenGrossenAgilen2015}[94]{hruschkaAgileSoftwareentwicklungGrossen2009}[2]{gollMitScrumGewuenschten2015}[S. 72 f.]{hanschkeAgilePlanungNur2016}

% \section{Einführung in das Projekt}
% Im Kern umfasst das Projekt die Erstellung eines Wissensquiz für den Raumplanungsassistenten \acs{RAPLA} in Moodle.
% Moodle ist ein Open-Source-Lernmanagementsystem, welches von vielen Bildungseinrichtungen weltweit eingesetzt wird.
% \footcite[Vgl.][S. 1 f.]{goepelDeutscheDatenschutzaspekteBei2008}
% Inhaltlich soll das Quiz Nutzerinnen und Nutzern Fragen zu den grundlegenden Funktionen und Eigenschaften von \acs{RAPLA} stellen, wobei
% hierbei auch Bilder und Videos verwendet werden können. Die Fragen sollen dabei in verschiedene Kategorien unterteilt werden, welche
% sich an den jeweiligen Funktionen der Anwendung orientieren. Grundsätzlich sollen die Fragen, welche
% sowohl als Multiple-Choice-Frage als auch mit Eingabeaufforderung abgebildet sein können, sich in ihrer Schwierigkeit unterscheiden.
% Der erfolgreiche Abschluss des Kurses bildet eine persönliche Zertifizierung, auf welcher Vor- und Nachname
% der Teilnehmerin oder des Teilnehmers sowie Datum und Ort der Zertifizerung vermerkt sind.
% Ferner hat das Zertifikat das Logo und den Namen der Hochschule sowie des Kurses zu enthalten. Die Zertifizierung soll in Form eines PDF-Dokuments
% zur Verfügung gestellt werden. Für die erfolgreiche Teilnahme am Kurs und den Erhalt des Zertifikats muss eine Mindestpunktzahl erreicht werden,
% welche sich aus der Summe der Punkte für alle Frage ergibt. Hierfür ist vorab 
% ein entsprechendes Schema anzufertigen. Bei Nichterreichen der Mindestpunktzahl soll eine Wiederholung des Kurses möglich sein.
% \section{Beschreibung der Ausgangssituation}
% Zum Zeitpunkt der Erstellung dieses Dokuments liegen diverse Schulungsdokumente für \acs{RAPLA} vor.
% Diese sind jedoch nicht einheitlich und unterscheiden sich sowohl inhaltlich als auch strukturell voneinander.
% Das Arbeitsteam hat die Möglichkeit, anhand des Moodle-Kurses in Statistik erste Erkennntnisse über die Abbildung
% von Schulungsinhalten in Moodle zu gewinnen. Dieser Kurs ist zwar inhaltlich nicht für die Erstellung eines Wissensquiz
% für \acs{RAPLA} konzipiert worden, kann jedoch als Grundlage für die Erstellung eines solchen Kurses dienen.
% Darüber hinaus lässt sich im Moodle-Kurs zur Projektkonzeption ein Testzertifikat finden, welches als Vorlage für das
% Zertifikat des Wissensquiz dienen kann. Hierzu liegen jedoch keine weiteren Informationen vor.
% \section{Beschreibung der Zielsetzung}
% Ziel des Projektes ist ein zusammenhängendes Wissensquiz für \acs{RAPLA} in Moodle, welches zehn bis 15 Fragen umfasst.
% Entsprechend der Anforderungen an das Projekt soll das Quiz in verschiedene Kategorien unterteilt sein, welche sich an den
% jeweiligen Möglichkeiten und Funktionen von \acs{RAPLA} orientieren. Der zeitliche Rahmen für die Bearbeitung des Kurses soll
% eine Dauer von 30 Minuten nicht überschreiten.
% \section{Anforderungen an die Systemtechnik}
% Als Benutzeroberfläche wird die Moodle-Instanz der DHBW Stuttgart verwendet. Dadurch wird der Anforderung, die Anwendung auf allen
% gängigen Betriebssystemen und Browsern verwenden zu können, Rechnung getragen. Die Anwendung soll sowohl auf Desktop-Computern als auch
% auf mobilen Endgeräten funktionieren.
% \section{Anforderungen an inhaltliche Aspekte}
% Inhaltlich soll das Quiz Nutzerinnen und Nutzern Fragen zu den grundlegenden Funktionen und Eigenschaften von \acs{RAPLA} stellen, welche
% sich in folgende Kategorien unterteilen lassen:
% \begin{itemize}
%     \item Grundlagen
%     \item Buchung von Räumen und Terminen
%     \item Verwalten gebuchter Termine
%     \item Erweiterte Funktionen
% \end{itemize}
% Genauere inhaltliche Anforderungen sind im Projektverlauf mit der Studiengangsleitung zu klären.
% \section{Anforderungen an die Qualität}
% Die Qualität des Projektes soll durch eine stetige Überprüfung und Anpassung der Anforderungen an das Projekt sichergestellt werden.
% Dies kann erreicht werden, indem einerseits die Anforderungen an das Projekt in regelmäßigen Abständen mit den
% anderen Projektmitgliedern besprochen und gegebenenfalls angepasst werden. Andererseits soll die Qualität des Projektes final durch eine Evaluation
% ermittelt werden, indem das Wissensquiz mit echten Anwenderinnen und Anwendern getestet wird. Hierbei soll vor allem die Benutzerfreundlichkeit
% des Kurses im Vordergrund stehen. Hierfür könnte bspw. ein Fragebogen erstellt werden, in welchem die Teilnehmerinnen und Teilnehmer Rückmeldung
% zum Kurs geben können. Für die inhaltliche Korrektheit der Fragen kann zuerst eine interne Überprüfung durch die Projektmitglieder erfolgen.
% In zweiter Instanz könnte im Kurs die Möglichkeit geschaffen werden, dass die Teilnehmerinnen und Teilnehmer Fragen, bei welchen sie den Einruck haben,
% dass eine falsche Antwort als richtig angegeben ist, melden können. Diese Fragen können dann von den Projektmitgliedern erneut überprüft und gegebenenfalls
% korrigiert werden.
% \section{Anforderungen an den Datenschutz}
% Da die Abbildung des Wissensquiz in einer integrierten DHBW-Instanz von Moodle erfolgen soll, gelten hierfür die Datenschutzbestimmungen der DHBW Stuttgart.
% Mit hoher Wahrscheinlichkeit ist hierfür eine Einwilligung der Teilnehmerinnen und Teilnehmer erforderlich, welche im Kurs vorab abgefragt werden muss.
% In einer solchen Erklärung soll unter anderem festgehalten sein, welche Daten zu welchem Zweck erhoben werden und wie lange diese gespeichert werden.
% Für genauere Informationen zur konkreten Ausgestaltung der Datenschutzerklärung ist der Datenschutzbeauftragte der DHBW Stuttgart zu kontaktieren.

