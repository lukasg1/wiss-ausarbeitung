\chapter{Konzeption des Wissensquiz}
\section{Analyse der vorhandenen Schulungsunterlagen}
\section{Erstellung und Aufbau des Fragenkatalogs}
Bei der Einteilung der Aufgaben wurde darauf geachtet, sowohl theoretische Konzepte, wie Definitionen, als auch praxisbezogene Fragen zu berücksichtigen. 
Die Fragen wurden in verschiedene Kategorien unterteilt, um eine bessere Übersicht und Struktur zu gewährleisten. 
Die Kategorien umfassen unter anderem:
\begin{enumerate}
    \item \textbf{Umsetzung des Theoreiteils:}
    \begin{itemize}
        \item Grundlegende Konzepte in der Verwendung von RAPLA
        \item Best-Practice-Beispiele für die effiziente Nutzung von RAPLA
        \item Definitionen von Begriffen und Funktionen in RAPLA
        \item Anbindung an Dualis
    \end{itemize}
    \item \textbf{Umsetzung des Praktischen Teils:}
    \begin{itemize}
        \item Umgang mit Filtern, Kalendern und Speichern sowie Exportieren
        \item Erstellung und Bearbeitung von Einzelterminen
        \item Erstellung und Bearbeitung von Serienterminen
        \item Einrichtung von Ausnahmen und Wiederholungen
        \item Fehlerhandling bei Konflikten
        \item Abarbeitung von komplexen Workflows
    \end{itemize}
\end{enumerate}
\section{Darlegung des Prüf- und Freigabeprozesses}
