\chapter{Abschließende Evaluation}
Abschließend kann festgestellt werden, dass zwar durch
einige kurzfristige Anforderungsänderungen Umplanungen notwendig
wurden, diese aber in Absprache mit dem Nachbar- sowie dem Entwicklerteam
innerhalb einer möglichst kurzen Zeit durchgeführt werden konnten, sodass
bereits Anfang des Jahres innerhalb der Projektgruppe der relevante Teil der Fragen geklärt war.

Für zukünftige Projekte sowie die Umsetzung der Zertifizierung im sechsten Semester ergeben sich einige
mögliche Implikationen. So bietet sich in folgenden Projekten die Diskussion einer agilen Arbeitsweise an,
da diese sich flexibler an Anforderungsänderungen anpassen lässt. Weiterhin sollten zeitliche Fristen enger
definiert werden, um zu verhindern, dass Fristen zwar eingehalten, aber in hohem Maße ausgereizt werden.
Darüber hinaus ist auch die Nutzung einer breiten Auswahl an verschiedenen Dokumentations-
und Kommunikationstools wie Jira, Word, WhatsApp und Discord eine mögliche Vergeudung im Sinne des Prozessmangements,
welche sich bspw. in Effizienzeinbußen und Kommunikationsproblemen äußern kann. Aus diesem Grund bietet sich auch hier
eine ausführliche Reflektion vor der Auswahl an Applikationen an. Als positiver Aspekt hat sich im Rahmen der Projektkonzeption
die problemlose Arbeitsteilung in der Gruppe erwiesen.

Die finale Evaluation der Planung lässt den Schluss zu, dass alle Schritte der Planung in der
geforderten Zeit durchgeführt wurden. Durch die detaillierte Anforderungs- und Risikoanalyse sowie
Umsetzungsplanung in Form von Arbeitspaketen ist auch für das folgende Semester eine im Zeitrahmen 
liegende Durchführbarkeit sichergestellt.