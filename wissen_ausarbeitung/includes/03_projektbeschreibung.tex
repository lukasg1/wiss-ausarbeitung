\chapter{Projektbeschreibung}
% noch mal Bezug nehmen auf das Projektplanungsdokument (nur das, was wirklich relevant
% für dieses Projekt ist)
\section{Ausgangslage und Problemstellung}
Die Ausgangslage dieses Projektes lässt sich durch die dringende Notwendigkeit einer umfassenden Schulung sowie einer anschließenden Zertifizierung zur Einführung des neuen Raumplanungsassistenten RAPLA beschreiben. Dieser Assistent soll den Prozess der Raumplanung erheblich vereinfachen und optimieren. Für die erfolgreiche Einführung ist es jedoch unerlässlich, dass die Nutzerinnen und Nutzer entsprechend geschult und zertifiziert werden. Im Rahmen dieser Gruppenarbeit liegt der organisatorische Schwerpunkt auf der Implementierung der Zertifizierung. Diese wird durch die Entwicklung eines Wissensquizzes und eines abschließenden Zertifizierungsquizzes realisiert. Das zugrunde liegende Projektplanungsdokument hebt hervor, dass die inhaltliche Komplexität des Raumplanungsassistenten RAPLA eine tiefgehende und umfassende Schulung sowie eine präzise Zertifizierung notwendig macht. Daher ist eine strukturierte und detaillierte Herangehensweise erforderlich, um sicherzustellen, dass alle relevanten Aspekte abgedeckt werden und die Nutzerinnen und Nutzer optimal vorbereitet sind.

\section{Anforderungen an das Quiz}
Die Anforderungen an das Quiz sind äußerst vielfältig und umfangreich. Zum einen soll das Quiz die Lernenden auf die bevorstehende Zertifizierung optimal vorbereiten, wobei ein besonderer Fokus auf einer hohen Benutzerfreundlichkeit liegt. Dies bedeutet, dass das Quiz intuitiv und einfach zu bedienen sein muss, um eine positive Lernerfahrung zu gewährleisten. Zum anderen dient das Quiz der Überprüfung des erworbenen Wissens in Bezug auf den gesamten Projektumfang. Hierbei ist es essenziell, dass das Quiz sowohl theoretische Fragen als auch praktische Aufgaben in unterschiedlichen Schwierigkeitsgraden enthält. Diese Fragen und Aufgaben müssen so formuliert sein, dass sie klare und eindeutige Antworten ermöglichen, was eine automatisierte Bewertung erleichtert. Nach der Bewertung soll den Lernenden eine Rückmeldung in Form eines Zertifikates gegeben werden, welches ihren Kenntnisstand offiziell bestätigt.
Für die Umsetzung des Quizzes wird die Lernplattform Moodle genutzt, da diese Plattform zahlreiche Funktionen bietet, die für die Erstellung und Durchführung eines interaktiven und effektiven Quizzes notwendig sind. Moodle ermöglicht es, verschiedene Fragetypen und Aufgabenformate zu integrieren, was zur Vielseitigkeit und Dynamik des Quizzes beiträgt. Darüber hinaus werden zwei reale Rapla-Instanzen zur Darstellung der Aufgaben verwendet. Insgesamt soll das Quiz nicht nur ein hohes Maß an Interaktivität bieten, sondern auch sicherstellen, dass die Lernenden intensiv mit den Inhalten des Raumplanungsassistenten RAPLA vertraut gemacht werden und so bestens auf die Zertifizierung vorbereitet sind.

\section{Methodik und Vorgehensweise}
Die Methodik und Vorgehensweise zur Umsetzung dieses Projektes ist in mehrere Phasen unterteilt, um eine systematische und strukturierte Herangehensweise zu gewährleisten.
In der Projektplanungsphase werden zunächst die Ziele und Aufgaben klar definiert. Ein detaillierter Zeitplan wird erstellt, der die verschiedenen Meilensteine des Projektes festlegt. Hierzu gehören unter anderem die Analyse, das Design, die Implementierung, das Testen und die finale Evaluierung.
Während der Analysephase wird eine umfassende Bedarfsanalyse durchgeführt, um die spezifischen Anforderungen an das Quiz zu ermitteln. Dies umfasst die Identifizierung der zu vermittelnden Inhalte sowie die Festlegung der Kriterien für die Zertifizierung. Die Anforderungsanalyse hilft dabei, die notwendigen Funktionalitäten und Eigenschaften des Quizzes zu bestimmen.
In der Designphase wird ein detailliertes Konzept für das Quiz entwickelt. Dies beinhaltet sowohl die inhaltliche Gestaltung als auch die Benutzeroberfläche. Das Ziel ist es, ein benutzerfreundliches und interaktives Quiz zu entwerfen, das den Lernenden eine effektive Vorbereitung ermöglicht.
Die Implementierungsphase umfasst die tatsächliche Programmierung des Quizzes. Dabei wird das Quiz in die Lernplattform Moodle integriert, die aufgrund ihrer vielseitigen Funktionen und Benutzerfreundlichkeit ausgewählt wurde. In dieser Phase werden die verschiedenen Fragetypen und Aufgabenformate erstellt und in das System eingebunden.
In der anschließenden Testphase wird das Quiz ausführlich getestet. Hierbei liegt der Fokus auf der Benutzerfreundlichkeit und der Funktionalität. Fehler und Probleme werden identifiziert und behoben, um sicherzustellen, dass das Quiz reibungslos funktioniert.
Die Evaluierung und Feedback-Phase beinhaltet das Sammeln von Rückmeldungen der ersten Nutzer. Basierend auf diesem Feedback werden notwendige Anpassungen vorgenommen, um die Qualität und Effektivität des Quizzes weiter zu verbessern.
In der letzten Phase, der Finalisierung und Rollout, wird die Abschlussdokumentation erstellt und das Quiz finalisiert. Zudem erfolgt der offizielle Rollout des Quizzes für alle Nutzer.
Diese strukturierte Vorgehensweise stellt sicher, dass das Projekt methodisch und effizient umgesetzt wird, wodurch die Ziele der Schulung und Zertifizierung des Raumplanungsassistenten RAPLA erfolgreich erreicht werden können.


