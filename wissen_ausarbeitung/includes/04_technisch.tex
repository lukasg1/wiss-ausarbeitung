\chapter{Technische Realisierung}
Gegenstand der Arbeitspaketeplanung ist die Struktur von Projekten hinsichtlich ihrer Komplexität.
\footcite[Vgl.][128]{kusterHandbuchProjektmanagementAgil2022}
Dies geschieht durch die Unterteilung des Projektes in kleinere Einheiten sowie durch die Erstellung eines Projektstrukturplans.
Oftmals wird dieser einschließlich seiner Meilensteine auf Arbeitspaketebene aufgeteilt und somit für Projektmitarbeiter editierbar gemacht.
\footcite[Vgl.][121]{panagosToolsUndMassnahmen2019}
Dies ist vergleichbar mit einer umgekehrten Baumstruktur, bei welcher im oberen Bereich der jeweilige Meilenstein dargestellt ist und die Blätter die
zugehörigen Arbeitspakete repräsentieren.
\footcite[Vgl.][29]{kusay-merkleAgilesProjektmanagementIm2018}
Im unteren Bereich sind schlussendlich die unteilbaren Arbeitspakete zu finden. In der Theorie können diese jedoch auf einer beliebigen Ebene liegen.
\footcite[Vgl.][73]{gadatschGrundkursITProjektcontrollingGrundlagen2008}
Das Finden von Arbeitspaketen findet hierbei ausgehend von der unteren Ebene statt.
Diese eignet sich insbesondere für Projektvorhaben, welchen eine geringere Erfahrung mit dem zu behandelnden Themengebiet zugrunde liegt.
Hierbei werden alle Tätigkeiten als Team zusammengeführt und anschließend Meilensteine mit den darunterliegenden Arbeitspaketen
definiert (siehe Abb. \ref{abb:psp}).
\footcite[Vgl.][133]{kusterHandbuchProjektmanagementAgil2022}
\begin{figure}[H]
    \centering
    \includegraphics[width=0.47\linewidth]{graphics/psp.png}
    \caption{Zuordnung von Arbeitspaketen innerhalb der Projektstrukturplanung.}\label{abb:psp}
\end{figure}
\section{Arbeitspaketebeschreibung und Aufwandsschätzung}
Bestandteile bei der Defintion von Arbeitspaketen sind die Leistungsbeschreibung, die jeweils verantwortliche Person
sowie die zugeordneten Ressourcen.
\footcite[Vgl.][73]{gadatschGrundkursITProjektcontrollingGrundlagen2008}
Zusätzlich können untergeordnete Aufgabenpakete mit entsprechenden Zeitangaben erstellt werden.
\footcite[Vgl.][74]{gadatschGrundkursITProjektcontrollingGrundlagen2008}
Als Ergebnis muss denoch schlussendlich ein minimal funktionsfähiges Produkt (in der Literatur: „\ac{MVP}“)
vorliegen, welches im Sinne des Wasserfall-Modells in die nächste Phase übernommen werden kann.
\footcite[Vgl.][52]{panagosToolsUndMassnahmen2019}
Als weitere Anforderung an die Arbeitspakete soll beim vorliegenden Projekt dafür gesorgt sein, dass die Ausführung
von einer einzelnen Person in einem angemessenen zeitlichen Rahmen bewältigt werden kann.
In Projekten, welche unter wirtschaftlichen Rahmenbedingungen stattfinden, werden für die Schätzung des Zeitbedarfs
der einzelnen Projektaktivitäten häufig \ac{PM} benutzt, bei welchen allerdings Vollzeitmitarbeiter als Grundlage
angesehen werden.
\footcite[Vgl.][74]{gadatschGrundkursITProjektcontrollingGrundlagen2008}
Da dieser Umstand bei einem Hochschulprojekt nicht gegeben ist, wird daher mit keiner konkret geschätzten Zeit
gearbeitet. Es wird hier nach interner Absprache zwischen dem Projektteam lediglich eine Abgabefrist für
Arbeitspakete gesetzt, welche für alle am Projekt beteiligten Personen einzuhalten ist. Weitere Informationen
hierzu sind in \textit{Kapitel 6} zu finden.
\section{Backlog und Akzeptanzkriterien für Arbeitspakete}
Das Backlog stellt eine Liste der gewünschten Arbeiten dar.
\footcite[Vgl.][362]{kusay-merkleAgilesProjektmanagementIm2018}
Diese noch zu erledigenden Aufgaben können dort bis zur abschließenden Nutzung gesammelt und aufbereitet werden.
\footcite[Vgl.][362]{kusay-merkleAgilesProjektmanagementIm2018}
Ferner werden die Arbeitspakete hierbei den großen Meilensteinen (in der Literatur: „Epics“) zugeordnet.
Als Werkzeug für diese Arbeitspakete- und Zeitplanung kommt das Programm „Jira Software“ zum Einsatz.
Es ist für kleine Projekte kostenlos nutzbar und enthält Funktionen für die Nachvollziehbarkeit von Aufgaben
und Fehlern sowie für die Verwaltung von Projekten.
\footcite[Vgl.][3]{ortuMeasuringUnderstandingEffectiveness2015}
Als vorteilhaft bei der Nutzung von Jira erweist sich der grundflexible Charakter der Software, welcher
es dem Projektteam unter anderem ermöglicht, Teilinkremente mit hoher Produktivität auszuliefern, das Backlog zu verwalten
sowie den Fortschritt des Projekts visuell darzustellen.
\footcite[Vgl.][3]{ortuMeasuringUnderstandingEffectiveness2015}
Die aktuellen Meilensteine im Backlog sind folgende:
\begin{enumerate}
    \item Ausarbeitung der Projektkonzeption
    \item Präsentation am Anfang des sechsten Semesters
    \item Interviews mit Leitfaden
    \item Wissenschaftliche Ausarbeitung im sechsten Semester
    \item Prototyp im sechsten Semester
    \item Ergebnispräsentation am Ende des sechsten Semesters
\end{enumerate}
Im Backlog werden schließlich für die einzelnen Arbeitspakete der Meilensteine Kriterien erstellt, anhand derer
erkenntlich wird, ob das jeweilige Objekt den Anforderungen entsprechend umgesetzt worden ist.
\footcite[Vgl.][46]{kusay-merkleAgilesProjektmanagementIm2018}
Vor der Fertigstellung eines Arbeitspakets müssen alle vorher definierten Akzeptanzkriterien erfüllt sein.
\footcite[Vgl.][154]{kusterHandbuchProjektmanagementAgil2022}