\chapter{Erprobung und Evaluation}
\section{Erprobung durch die Zielgruppe}
Zur Sicherstellung einer hohen Qualität des Wissensquiz wurde dieses intensiv durch die Zielgruppe erprobt. Die Zielgruppe besteht aus drei erfahrenen Sekretärinnen, die über unterschiedliche technische Affinitäten verfügen sowie verschiedene Grade der RAPLA-Vorerfahrung mitbringen. Diese Unterschiede innerhalb der Zielgruppe ermöglichen eine breit gefächerte Bewertung und garantieren, dass das Feedback aus verschiedenen Perspektiven gegeben wird. Die Erprobung erfolgte in Präsenz in der DHBW-Stuttgart, wo die Probandinnen an ihrem eigenen Rechner das Wissensquiz bearbeiteten und dabei sowohl die technische Umsetzung als auch die inhaltliche Korrektheit sorgfältig bewerteten. Zusätzlich zu diesen Aspekten lag ein weiterer Fokus auf der Evaluation des Komplexitätsgrades des Quiz, um sicherzustellen, dass es für alle Zielgruppenmitglieder angemessen und herausfordernd ist.
Die Probandinnen wurden gebeten, das Wissensquiz vollständig zu bearbeiten und anschließend ein ausführliches Feedback zu geben. Um ein umfassendes Bild zu erhalten, wurde sich dabei an den folgenden Kernfragen orientiert, die sowohl technische als auch inhaltliche Aspekte des Quiz abdeckten. Dies ermöglichte es, eine detaillierte Analyse der Nutzererfahrungen zu erstellen und gezielte Verbesserungsvorschläge zu erarbeiten.
Darüber hinaus wurde regelmäßig von Herrn Kohlhaas Feedback zu den Fragen und Aufgaben eingeholt.

\begin{enumerate}
    \item \textbf{Technische Umsetzung:}
    \begin{enumerate}
        \item Ist die technische Umsetzung funktional und benutzerfreundlich oder gibt es Herausforderungen?
        \item Wie intuitiv ist die Navigation durch das Quiz?
        \item Treten während der Nutzung technische Probleme oder Fehler auf?
        \item Ist die Ladezeit der einzelnen Fragen und der Ergebnisse akzeptabel?
    \end{enumerate}
    \item \textbf{Inhaltliche Bewertung:}
    \begin{enumerate}
        \item Ist der Inhalt des Wissensquiz verständlich und korrekt oder gibt es Unklarheiten?
        \item Sind die Fragen klar formuliert und leicht verständlich?
        \item Decken die Fragen alle relevanten Themengebiete ab?
        \item Gibt es inhaltliche Fehler oder missverständliche Formulierungen?
    \end{enumerate}
    \item \textbf{Komplexitätsgrad:}
    \begin{enumerate}
        \item Ist der inhaltliche Komplexitätsgrad des Wissensquiz (für neue Mitarbeitende) angemessen oder zu hoch bzw. zu niedrig?
        \item Sind die Fragen zu einfach oder zu schwierig für den vorgesehenen Zweck?
        \item Wird der Wissensstand der Zielgruppe angemessen berücksichtigt?
    \end{enumerate}
\end{enumerate}

Die Erprobung wurde in drei Phasen durchgeführt:

\begin{enumerate}
    \item \textbf{Einführungsphase:}
    \begin{itemize}
        \item In dieser Phase wurde den Probandinnen eine kurze Einführung in die Nutzung des Quiz und dessen Zielsetzung gegeben. Es wurde erläutert, welche Aspekte besonders im Fokus der Evaluation stehen.
    \end{itemize}
    \item \textbf{Durchführungsphase:}
    \begin{itemize}
        \item Die Probandinnen bearbeiteten das Wissensquiz individuell an ihren eigenen Arbeitsplätzen. Dabei wurde darauf geachtet, dass sie sich in einer möglichst realitätsnahen Umgebung befanden, um authentische Rückmeldungen zu gewährleisten.
    \end{itemize}
    \item \textbf{Feedback-Phase:}
    \begin{itemize}
        \item Nach Abschluss der Quizbearbeitung wurde eine Feedbackrunde durchgeführt. Jede Probandin wurde zu den oben beschriebenen Kernfragen befragt.
        \item Zusätzlich wurden die Probandinnen dazu eingeladen, ihre Erfahrungen und Verbesserungsvorschläge jederzeit mündlich oder auch als Nachreichung schriftlich zu äußern.
    \end{itemize}
\end{enumerate}
Diese umfassende Erprobung durch die Zielgruppe bildet die Grundlage für die anschließende Analyse der 
Erprobungsresultate und die Ableitung von Optimierungsmaßnahmen, um das Wissensquiz weiter zu verbessern 
und an die Bedürfnisse der Zielgruppe anzupassen.

\section{Analyse der Erprobungsresultate}
Das im Rahmen der Erprobung gesammelte Feedback wurde anschließend detailliert ausgewertet, um Stärken und 
Schwächen des Wissensquiz zu identifizieren. Dabei wurden überwiegend qualitative Rückmeldungen 
(z.B. subjektive Einschätzungen der Benutzerfreundlichkeit und Verständlichkeit oder auch Hinweise zu spezifischen Fragen) 
berücksichtigt.

Die Erprobung des Wissensquiz durch die Zielgruppe ergab eine Vielzahl von wertvollen Erkenntnissen 
und Verbesserungsvorschlägen. 
Die Probandinnen bewerteten die technische Umsetzung des Quiz insgesamt 
als funktional und benutzerfreundlich. Die Navigation durch das Quiz wurde als intuitiv und übersichtlich 
empfunden, sodass die Probandinnen keine Schwierigkeiten hatten, sich zurechtzufinden. 
Die relevantesten  Punkte, die im Rahmen der Erprobung identifiziert wurden, sind:

\begin{itemize}
    \item \textbf{Allgemeine Hinweise:}
    \begin{itemize}
        \item Es wurde angemerkt, dass in der Anleitung viele Hinweise leicht zu überlesen waren. 
        \item Einige Informationen zum Ablauf des Quiz und zum Umgang mit den Rapla-Instanzen sollten ergänzt werden, wie beispielsweise ein Hinweis darauf, dass nach der Gruppeneinschreibung RAPLA idealerweise direkt geöffnet werden sollte.
    \end{itemize}
        \item \textbf{Inhaltliche Umsetzung der Theoriefragen:}
    \begin{itemize}
        \item Einige Probandinnen bemängelten, dass die Fragen teilweise zu theoretisch und abstrakt formuliert sind und zu wenig Praxisbezug gegeben war.
        \item Die Fragen wiesen teilweise sprachliche Schwächen in der Orthographie und Grammatik auf.
    \end{itemize}
    \item \textbf{Inhaltliche Umsetzung des Praxisteils:}
    \begin{itemize}
        \item Die Probandinnen lobten die praxisnahen Fragen und die realitätsnahe Darstellung der Aufgabenstellungen.
        \item Wenige Fragen waren jedoch unklar formuliert, weswegen entweder mit Räumen oder Personen improvisiert werden musste oder mehrere Antwortmöglichkeiten korrekt waren.
        \item Wenige Fragen bezogen sich auf Vorgehensweisen aus der Dokumentation sowie dem Handbuch und gaben hierdurch andere Wege vor als von Testerinnen präferiert wurde.
        \item In einigen Fällen waren die vorgegebenen Antworten nicht korrekt.
    \end{itemize}
\end{itemize}
\section{Ableitung von Optimierungsmaßnahmen}
Die Analyse der Erprobungsresultate hat gezeigt, dass das Wissensquiz insgesamt gut angenommen wurde, 
jedoch noch einige Optimierungen und Anpassungen erforderlich sind, um den Anforderungen der Zielgruppe
gerecht zu werden. So wurden die folgenden Maßnahmen zur Verbesserung des Wissensquiz abgeleitet:

\begin{itemize}
    \item \textbf{Überarbeitung der Anleitung:}
    \begin{itemize}
        \item Die Anleitung des Wissensquiz wird überarbeitet und um zusätzliche Hinweise und Erläuterungen ergänzt, um die Benutzerfreundlichkeit zu erhöhen.
        \item Die Anleitung wird klar strukturiert und übersichtlich gestaltet, um wichtige Informationen durch Erhöhung der Schriftstärke hervorzuheben und leichter auffindbar zu machen.
    \end{itemize}
    \item \textbf{Überarbeitung der Fragen:}
        \item Die Fragen des Wissensquiz wurden teilweise überarbeitet und praxisnäher formuliert, um den Bezug zur realen Arbeitswelt zu stärken.
        \item Fehler in der Rechtschreibung und Grammatik wurden korrigiert, um die Verständlichkeit der Fragen zu verbessern.
        \item In einem erneuten Testlauf ohne die Probandinnnen wurden die Fragen kontrolliert und gegebenenfalls die möglichen Antworten angepasst.
\end{itemize}

\section{Weitere Optimierungsmaßnahmen}
Da aus Gründen der Zeit ausschließlich eine Erprobung des Wissensquiz durch die Zielgruppe durchgeführt wurde,
ergab sich eine Notwendigkeit für eine weitere Evalation der Zertifizierung.
Zum Einen wurde hierzu bereits von Herrn Kohlhaas Feedback gegeben, welches überwiegend positiv ausfiel und
die Komplexität der Fragen lobte. Zum Anderen musste die Praxis der Zertifizierung jedoch einmal getestet wurden.
Hierfür wurde ein Tag eingeplant, an welchem alle Fragen und Aufgaben bearbeitet und mit den Musterlösungen abgeglichen wurden.
Dabei fiel auf, dass auch hier Fehler in der Orthopraphie und Grammatik sowie missverständliche Formulierungen auftraten, welche verbessert werden mussten.
Zudem zeigte sich auch hier die Notwendigkeit einer erneuten Überprüfung, denn nicht alle Musterantworten waren korrekt. 
Alle gefundenen Fehler wurden korrigiert und die Zeritifizierung erneut durchgeführt.


