\chapter{Zeitplanung}
Die Zeitplanung spielt im Kontext des Hochschulprojektes eine bedeutende Rolle,
da das Projekt zeitlich eng begrenzt und auf zwei verschiedene Semester aufgeteilt ist.
Es ergibt sich die Notwendigkeit einer detaillierten Planung, um alle Anforderungen
termingerecht erfüllen zu können und Abhängigkeiten zwischen mehreren Teammitgliedern
konfliktfrei zu lösen. Weiterhin müssen separate Planungen für die jeweiligen Semester
durchgeführt werden, da diese zwar abhängig voneinander sind, die Informationen für die
Planung des höheren Semesters allerdings noch nicht vorhanden sein müssen.

Die Planung dazu basiert - wie in vorherigen Kapiteln beschrieben - allgemein auf
festen Abgabeterminen zu bereits festgelegten Meetings. 
Begonnen wurde mit der groben Definition von Epics
und Arbeitspakten in Jira, auf deren Basis dann eine vorläufige Planung für beide Semester
erstellt wurde. Diese konnte schrittweise in ihrem Detailgrad ausgebaut werden.

Zur Visualisierung wurde die Zeitleiste in Jira verwendet, die einem Gantt-Diagramm ähnelt.
Das Gantt-Diagramm bietet sich bei der Zeitplanung hierbei besonders an, da es eine
übersichtliche Möglichkeit bietet, Verknüpfungen mit vorangehenden Vorgängen,
Dokumentation der Vorgangsverantwortlichen sowie Kapazitätsdarstellungen, in einem
Diagramm zu veranschaulichen.
\footcite[Vgl.][123]{osterhageAnhangProjektmanagement2016}
Dadurch, dass über die Zeitachse Aktivitäten in Form von Balken mit festen Anfangs-
und Endtermin dargestellt werden, können Abhängigkeiten zwischen Aufgaben und
Arbeitspaketen hergestellt sowie kritische Stellen in der Planung sichtbar gemacht
werden.
\footcite[Vgl.][117]{hobelGABLERBUSINESSWISSENAZ2006}
Zudem erklärt sich die populäre Nutzung von Gantt-Charts auch dadurch, dass
sie sowohl den Projektteilnehmern eine übersichtliche Visualisierung des Projektfortschritts
ermöglichen als auch in der finalen Präsentation der Ergebnisse vor dem Management genutzt
werden können.
\footcite[Vgl.][435]{wilsonGanttChartsCentenary2003}
\begin{figure}[H]
    \centering
    \includegraphics[width=0.9\linewidth]{graphics/zeitplanung.png}
    \caption{Übersicht über die Zeitplanung in JIRA.}\label{abb:zeitplanung}
\end{figure}
Es wird sichtbar, dass die Kapitel nach einer logischen Reihenfolge eingeteilt worden sind.
Allgemeine Planungsthemen wie Projektplanung oder
Zeitplanung sind auf einen größeren Zeitraum zugeschnitten, konkrete Kapitelthemen wie die
Einleitung oder auch die Anforderungen dagegen für kurze und spezifische Zeitperioden vorgesehen
sowie in einer für den Ablauf effizienten, sequenziellen Reihenfolge angeordnet. Hieraus ergibt
sich, dass das sechste Semester überwiegend für die Umsetzung genutzt werden kann und jegliche
Planung bereits zu großen Teilen finalisiert ist.

Bezüglich der Umsetzungsplanung lässt sich feststellen, dass die Umsetzung der Schulungsunterlage
einige Abhängigkeiten mit sich bringt. So erfordert sie enge Absprachen mit dem zweiten
Schulungs-Team, um keine Dopplungen zu erhalten. Diese Planung kann sich jedoch aufgrund
des neuen Vorlesungsplans im sechsten Semester noch ändern. 

